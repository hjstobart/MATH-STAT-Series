\documentclass[11pt]{article}
\usepackage[margin=1.2in]{geometry} 
\usepackage{amsmath}
\usepackage{tcolorbox}
\usepackage{amssymb}
\usepackage{amsthm}
\usepackage{lastpage}
\usepackage{fancyhdr}
\usepackage{accents}
\usepackage{parskip}
\pagestyle{fancy}
\setlength{\headheight}{40pt}

\begin{document}

\lhead{Mr. \textsc{H. Stobart}} 
\rhead{\textsc{Math/Stat Series \\ Issue 2, Feb-22}}
\cfoot{\thepage\ of \pageref{LastPage}}

\begin{tcolorbox}
\begin{center}
    \large
    \textsc{Solving Ordinary Differential Equations Using \\ An Integrating Factor}
\end{center}
\end{tcolorbox}

\begin{center}
\textbf{Note:} \textit{This work is intended for informative and educational purposes only.}
\end{center}

\section*{1. Introduction}
Last month we looked at one approach for solving ODES. In this month's issue, we continue in the same vein and consider how to solve one-dimensional ODE's using the integrating factor technique. This method has far reaching applications in fields such as Physics, Chemistry, Biology, and Finance.

\section*{2. Setup}
Suppose we seek the function $y(x) := y: \mathbb{R} \longrightarrow \mathbb{R}$, that satisfies the following ODE,
\begin{equation}
    \frac{dy}{dx} + g(x)y = h(x).
\end{equation}
Where $g(x)$ and $h(x)$ are arbitrary functions of $x$ only. Then we can use the integrating factor technique to solve the ODE. 

\section*{3. Method}
To solve this equation we will make use of the product rule from ordinary calculus. Before doing so, however, let us introduce a new term,
\begin{equation}
    e^{G(x)} := e^{\int g(x) dx}.
\end{equation}
Clearly, $g(x) = G'(x)$. This term is called the \textbf{Integrating Factor}.

By introducing this term we are able to take advantage of the results from differentiating exponential functions. Now, let us multiply our ODE (1) by this new term.
\begin{equation}
    e^{G(x)} \frac{dy}{dx} + g(x) e^{G(x)} y = h(x) e^{G(x)}.
\end{equation}
Examining the first two terms of this equation, we can see that this is precisely the product rule applied to the function $ye^{G(x)}$. That was lucky!
Let's rewrite this equation once more,
\begin{equation}
    \frac{d}{dx} \left[ ye^{G(x)} \right] = h(x)e^{G(x)}.
\end{equation}
This looks much more familiar. Let's integrate both sides.
\begin{equation}
    \int \frac{d}{dx} \left[ ye^{G(x)} \right] dx = \int h(x)e^{G(x)}.
\end{equation}
But we know that an integral and a derivative `cancel out', so finally we have,
\begin{equation}
    y(x) = e^{-G(x)} \int h(x)e^{G(x)}.
\end{equation}

\section*{4. Example}
Let's consider a specific example. Find the general solution of the following ODE,
\begin{equation}
    \frac{dy}{dx} + 3x^2 y = e^{-x^3}.
\end{equation}
Let's start by working out the integrating factor,
\begin{equation}
    e^{\int 3x^2 dx} = e^{x^3}.
\end{equation}
So, multiplying through our ODE becomes,
\begin{equation}
    e^{x^3} \frac{dy}{dx} + 3x^2 e^{x^3} y = 1.
\end{equation}
Which we can manipulate using the product rule,
\begin{equation}
    \int \frac{d}{dx} \left[ e^{x^3} y \right] dx = \int 1 dx.
\end{equation}
Which becomes,
\begin{equation}
    e^{x^3} y + c_1 = x + c_2.
\end{equation}
And finally, we have our solution,
\begin{equation}
    y = \frac{x + C}{e^{x^3}}.
\end{equation}
    
\end{document}
