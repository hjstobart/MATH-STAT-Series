\documentclass[11pt]{article}
\usepackage[margin=1.2in]{geometry} 
\usepackage{amsmath}
\usepackage{tcolorbox}
\usepackage{amssymb}
\usepackage{amsthm}
\usepackage{lastpage}
\usepackage{fancyhdr}
\usepackage{accents}
\pagestyle{fancy}
\setlength{\headheight}{40pt}
%\linespread{1.3}


\newenvironment{solution}
  {\renewcommand\qedsymbol{$\blacksquare$}
  \begin{proof}[Solution]}
  {\end{proof}}
\renewcommand\qedsymbol{$\blacksquare$}

\newcommand{\ubar}[1]{\underaccent{\bar}{#1}}
\newcommand{\HRule}{\rule{\textwidth{1mm}}} % Defines a new command for horizontal lines, change thickness here


\begin{document}

\lhead{Mr. \textsc{H. Stobart}} 
\rhead{\textsc{Math/Stat Series \\ Issue 1, May-22}}
\cfoot{\thepage\ of \pageref{LastPage}}

\begin{tcolorbox}
\begin{center}
    \large
    \textsc{Solving Ordinary Differential Equations Using Separation of Variables}
\end{center}
\end{tcolorbox}

\begin{center}
\textbf{Note:} \textit{This work is intended for informative and educational purposes only.}
\end{center}

\section*{1. Introduction}
We consider how to solve one-dimensional ordinary differential equations (ODEs) using the separation of variables technique. This is, perhaps, the most straightforward of techniques for solving ODEs, we shall explore different techniques in other issues.

\section*{2. Setup}
Suppose we have two functions $f: \mathbb{R} \longrightarrow \mathbb{R}$, and $g: \mathbb{R} \longrightarrow \mathbb{R}$. Also, suppose we can write our ODE in the following form.
\begin{equation}
    \frac{df(x)}{dx} = g(x)h(f(x))
\end{equation}
Then we can use the separation of variables technique to solve the ODE. 

\section*{3. Method}
We start by slightly modifying (1) for aesthetic purposes only. Letting $y = f(x)$, we write,
\begin{equation}
    \frac{dy}{dx} = g(x)h(y).
\end{equation}
Then under the condition that $h(y) \neq 0$, we can divide through by $h(y)$. This gives,
\begin{equation}
    \frac{1}{h(y)} \frac{dy}{dx} = g(x).
\end{equation}
Now multiplying through by $dx$ and integrating we obtain,
\begin{equation}
   \int \frac{1}{h(y)} \frac{dy}{dx} dx = \int g(x) dx .
\end{equation}
Which simply becomes,
\begin{equation}
   \int \frac{1}{h(y)} dy = \int g(x) dx .
\end{equation}
Thus solving our ODE. It is clear where the term \textit{separation of variables} comes from, as we are quite literally separating the variables either side of the equals sign. $\square$

\newpage

\section*{4. Example}
Let's look at a specific example. Find the general solution of the following ODE,
\begin{equation}
    \frac{dy}{dx} = y^2 e^{-4x}.
\end{equation}
We can clearly see that $g(x) = e^{-4x}$ and $h(y) = y^2$. Separating the variables we find,
\begin{equation}
    \int \frac{1}{y^2} \frac{dy}{dx} dx = \int e^{-4x} dx.
\end{equation}
Which simplifies to,
\begin{equation}
    \int \frac{1}{y^2} dy = \int e^{-4x} dx.
\end{equation}
Integrating both sides we obtain,
\begin{equation}
    -\frac{1}{y} + c_1 = -\frac{1}{4 }e^{-4x} + c_2.
\end{equation}
Which gives, 
\begin{equation}
    \frac{1}{y} = \frac{1 + c_3 e^{4x}}{4e^{4x}}.
\end{equation}
Finally, we find the solution,
\begin{equation}
   y = \frac{4e^{4x}}{1 + c_3 e^{4x}}.
\end{equation}
Unfortunately, in this example the constants of integration make the solution appear a somewhat convoluted expression, however, it does serve to illustrate their importance. 
\end{document}
